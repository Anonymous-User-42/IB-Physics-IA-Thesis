
        
\begin{tikzpicture}[H]
            \begin{axis}[
            scatter/classes={
%a={mark=square*,blue},
%b={mark=triangle*,red},
%c={mark=o,draw=black} % <-- don't add comma
c={mark=*,draw=green}
},
%                title={\textit{Graph in relation to the \textbf{experimental} \& \textbf{simulation} values of \textbf{drag force} versus \textbf{temperature}}},
                xlabel={\textbf{\textit{Temperature (in $\bm{^\circ}$C) [T]}}},
                ylabel={\textbf{\textit{Drag Force (in $\bm{10^{-3}}$ N) [$\bm{F_D}$]}}},
                xmin=0, xmax=100,
                ymin=4, ymax=5,
                xtick={10,20,30,40,50,60,70,80,90},
                ytick={4.1,4.2,4.3,4.4,4.5,4.6,4.7,4.8,4.9},
                legend pos=north east,
                ymajorgrids=true,
                xmajorgrids=true,
                grid style=dashed,
                legend entries={\textit{Regression Line} ($-0.000024x^2 - 0.001179x + 4.61738$),\textit{Drag Force} (\textit{Experimental})}
            ]
                
%            \caption{\textit{Graph in relation to the \textbf{experimental} \& \textbf{simulation} values of \textbf{drag force} versus \textbf{temperature}}}

%\begin{axis}

			\addplot+[color=magenta,domain=10:90,samples=100, mark=none]{(-0.000024)*x^2 + (-0.001179)*x + (4.61738)};

%\end{axis}

% \addplot[] is better than \addplot+[] here:
% it avoids scalings of the cycle list
\addplot [
scatter,only marks,
scatter src=explicit symbolic,
] coordinates {
(10,4.600) [c]
(10,4.630) [c]
(10,4.650) [c]
(10,4.570) [c]
(10,4.550) [c]
(20,4.590) [c]
(20,4.613) [c]
(20,4.635) [c]
(20,4.568) [c]
(20,4.547) [c]
(30,4.560) [c]
(30,4.593) [c]
(30,4.540) [c]
(30,4.508) [c]
(30,4.617) [c]
(40,4.530) [c]
(40,4.563) [c]
(40,4.585) [c]
(40,4.468) [c]
(40,4.489) [c]
(50,4.500) [c]
(50,4.533) [c]
(50,4.555) [c]
(50,4.478) [c]
(50,4.449) [c]
(60,4.460) [c]
(60,4.483) [c]
(60,4.515) [c]
(60,4.438) [c]
(60,4.407) [c]
(70,4.420) [c]
(70,4.453) [c]
(70,4.475) [c]
(70,4.398) [c]
(70,4.377) [c]
(80,4.370) [c]
(80,4.403) [c]
(80,4.425) [c]
(80,4.328) [c]
(80,4.307) [c]
(90,4.320) [c]
(90,4.353) [c]
(90,4.375) [c]
(90,4.278) [c]
(90,4.257) [c]
};

		\end{axis}
			
        \end{tikzpicture}

{The above graph is the \textbf{scatter plot} of all \textbf{data points} within the \textbf{data set} of \textbf{experimentally collected values} with the \textbf{quadratic regression} of the same. A \textbf{quadratic regression} was utilized to plot the \textbf{line of best fit}, because a regression creates an \textbf{accurate model} of the system according to the data points and is a \textbf{mathematical function of each and every data point}. This was used rather than plotting an simple line between the \textbf{range} of the \textbf{maximum and minimum} values of the data points, as it would be similar to guesswork.}

\begin{tikzpicture}[H]
            \begin{axis}[
            scatter/classes={
%a={mark=square*,blue},
%b={mark=triangle*,red},
%c={mark=o,draw=black} % <-- don't add comma
c={mark=*,draw=green}
},
%                title={\textit{Graph in relation to the \textbf{experimental} \& \textbf{simulation} values of \textbf{drag force} versus \textbf{temperature}}},
                xlabel={\textbf{\textit{Temperature (in $\bm{^\circ}$C) [T]}}},
                ylabel={\textbf{\textit{Drag Force (in $\bm{10^{-3}}$ N) [$\bm{F_D}$]}}},
                xmin=0, xmax=100,
                ymin=4, ymax=5,
                xtick={10,20,30,40,50,60,70,80,90},
                ytick={4.1,4.2,4.3,4.4,4.5,4.6,4.7,4.8,4.9},
                legend pos=north east,
                ymajorgrids=true,
                xmajorgrids=true,
                grid style=dashed,
                legend entries={$F_D$ (\textit{Simulation})}
            ]
                
            \addplot+[
                color=blue,
                mark=square,
                ] plot[error bars/.cd, y dir=both, y explicit, x dir=both, x explicit]
                coordinates {
                (10,4.60) +- (2,0.05)
                (20,4.59) +- (2,0.05)
                (30,4.56) +- (2,0.05)
                (40,4.53) +- (2,0.05)
                (50,4.50) +- (2,0.05)
                (60,4.46) +- (2,0.05)
                (70,4.42) +- (2,0.05)
                (80,4.37) +- (2,0.05)
                (90,4.32) +- (2,0.05)
                };
                                
                
%            \end{axis}
%            \caption{\textit{Graph in relation to the \textbf{experimental} \& \textbf{simulation} values of \textbf{drag force} versus \textbf{temperature}}}

%\begin{axis}

%			\addplot+[color=green,domain=0:100,samples=1000]{(-0.000024)*x^2 + (-0.001179)*x + (4.61738)};

%\end{axis}



		\end{axis}
			
        \end{tikzpicture}

{The above graph is the collection of single data points at specific intervals created with accordance to a simulation, based accurately on the \textbf{mathematical model} with \textbf{zero uncertainty}. On observation it looks evidently \textbf{similar to the quadratic regression line} from the previous graph, this is an \textbf{evidence that the experimental data collected was of least uncertainty and least errors} to an extent that was possible. The above graph also contains \textbf{ranges of values }in the form of \textbf{error bars} that show that the data points can range from a particular minimum to a particular maximum.}

%\begin{tikzpicture}
%\begin{axis}[
%scatter/classes={
%%a={mark=square*,blue},
%%b={mark=triangle*,red},
%%c={mark=o,draw=black} % <-- don't add comma
%c={mark=*,draw=green}
%},
%                xlabel={\textbf{\textit{Temperature (in $\bm{^\circ}$C) [T]}}},
%                ylabel={\textbf{\textit{Drag Force (in $\bm{10^{-3}}$ N) [$\bm{F_D}$]}}},
%                xmin=0, xmax=100,
%                ymin=4, ymax=5,
%                xtick={10,20,30,40,50,60,70,80,90},
%                ytick={4.1,4.2,4.3,4.4,4.5,4.6,4.7,4.8,4.9},
%                legend pos=north east,
%                ymajorgrids=true,
%                xmajorgrids=true,
%                grid style=dashed,
%                legend entries={$F_D$ (\textit{Experimental}), $F_D$ (\textit{Simulation})}
%]
%% \addplot[] is better than \addplot+[] here:
%% it avoids scalings of the cycle list
%\addplot [
%scatter,only marks,
%scatter src=explicit symbolic,
%] coordinates {
%(10,4.600) [c]
%(10,4.630) [c]
%(10,4.650) [c]
%(10,4.570) [c]
%(10,4.550) [c]
%(20,4.590) [c]
%(20,4.613) [c]
%(20,4.635) [c]
%(20,4.568) [c]
%(20,4.547) [c]
%(30,4.560) [c]
%(30,4.593) [c]
%(30,4.540) [c]
%(30,4.508) [c]
%(30,4.617) [c]
%(40,4.530) [c]
%(40,4.563) [c]
%(40,4.585) [c]
%(40,4.468) [c]
%(40,4.489) [c]
%(50,4.500) [c]
%(50,4.533) [c]
%(50,4.555) [c]
%(50,4.478) [c]
%(50,4.449) [c]
%(60,4.460) [c]
%(60,4.483) [c]
%(60,4.515) [c]
%(60,4.438) [c]
%(60,4.407) [c]
%(70,4.420) [c]
%(70,4.453) [c]
%(70,4.475) [c]
%(70,4.398) [c]
%(70,4.377) [c]
%(80,4.370) [c]
%(80,4.403) [c]
%(80,4.425) [c]
%(80,4.328) [c]
%(80,4.307) [c]
%(90,4.320) [c]
%(90,4.353) [c]
%(90,4.375) [c]
%(90,4.278) [c]
%(90,4.257) [c]
%};
%\end{axis}
%\end{tikzpicture}

\begin{tikzpicture}[H]
            \begin{axis}[
            scatter/classes={
%a={mark=square*,blue},
%b={mark=triangle*,red},
%c={mark=o,draw=black} % <-- don't add comma
c={mark=*,draw=green}
},
%                title={\textit{Graph in relation to the \textbf{experimental} \& \textbf{simulation} values of \textbf{drag force} versus \textbf{temperature}}},
                xlabel={\textbf{\textit{Temperature (in $\bm{^\circ}$C) [T]}}},
                ylabel={\textbf{\textit{Drag Force (in $\bm{10^{-3}}$ N) [$\bm{F_D}$]}}},
                xmin=0, xmax=100,
                ymin=4, ymax=5,
                xtick={10,20,30,40,50,60,70,80,90},
                ytick={4.1,4.2,4.3,4.4,4.5,4.6,4.7,4.8,4.9},
                legend pos=north east,
                ymajorgrids=true,
                xmajorgrids=true,
                grid style=dashed,
                legend entries={$F_D$ (\textit{Simulation}),$F_D$ (\textit{Experimental})}
            ]
                
            \addplot+[
                color=blue,
                mark=square,
                ] plot[error bars/.cd, y dir=both, y explicit, x dir=both, x explicit]
                coordinates {
                (10,4.60) +- (2,0.05)
                (20,4.59) +- (2,0.05)
                (30,4.56) +- (2,0.05)
                (40,4.53) +- (2,0.05)
                (50,4.50) +- (2,0.05)
                (60,4.46) +- (2,0.05)
                (70,4.42) +- (2,0.05)
                (80,4.37) +- (2,0.05)
                (90,4.32) +- (2,0.05)
                };
                                
                
%            \end{axis}
%            \caption{\textit{Graph in relation to the \textbf{experimental} \& \textbf{simulation} values of \textbf{drag force} versus \textbf{temperature}}}

%\begin{axis}

%			\addplot+[color=green,domain=0:100,samples=100]{(3.69*(802.6*(1-0.0011*x)))/10^3};

%\end{axis}

\addplot [
scatter,only marks,
scatter src=explicit symbolic,
] coordinates {
(10,4.600) [c]
(10,4.630) [c]
(10,4.650) [c]
(10,4.570) [c]
(10,4.550) [c]
(20,4.590) [c]
(20,4.613) [c]
(20,4.635) [c]
(20,4.568) [c]
(20,4.547) [c]
(30,4.560) [c]
(30,4.593) [c]
(30,4.540) [c]
(30,4.508) [c]
(30,4.617) [c]
(40,4.530) [c]
(40,4.563) [c]
(40,4.585) [c]
(40,4.468) [c]
(40,4.489) [c]
(50,4.500) [c]
(50,4.533) [c]
(50,4.555) [c]
(50,4.478) [c]
(50,4.449) [c]
(60,4.460) [c]
(60,4.483) [c]
(60,4.515) [c]
(60,4.438) [c]
(60,4.407) [c]
(70,4.420) [c]
(70,4.453) [c]
(70,4.475) [c]
(70,4.398) [c]
(70,4.377) [c]
(80,4.370) [c]
(80,4.403) [c]
(80,4.425) [c]
(80,4.328) [c]
(80,4.307) [c]
(90,4.320) [c]
(90,4.353) [c]
(90,4.375) [c]
(90,4.278) [c]
(90,4.257) [c]
};

		\end{axis}
			
        \end{tikzpicture}

{The above graph is a \textbf{culmination} of all the \textbf{experimentally collected data points} and the \textbf{data points collected via a simulation}. The \textbf{matching similarity} of the underlying graphs shows that the data is of much \textbf{reliability} and \textbf{devoid of errors}. According to the graph from the simulation, the error bars range lower than the maximum and minimum values (for most cases) of the experimentally collected values, indicating that such values are \textbf{outliers} and indicates that there has been \textbf{some errors and uncertainties} in the collection of data.}
